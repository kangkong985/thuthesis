\chapter{总结与展望}
\label{cha:conclusion}

\section{工作总结}
回顾全文,首先在第一章介绍了项目的背景和意义,无线Mesh网络的快速发展对人们的生产生活所产生
的影响越来越大,尤其在一切基础网络设施缺乏的特殊场合,比如灾后重建、野外大规模作业等,Mesh
网络能够充分发挥其易部署、低成本、高可靠性的特点,为这类工作提供很好的支持。然而Mesh网络
因为其特殊的网络构建方式和无线信道的开发性,导致难以提供QoS保障。进而第二章介绍了相关
领域的研究现状,以及目前Mesh网络的主流协议。

在此基础上,第三章提出了文章三个核心的创新点:\textbf{子网信道隔离的部署方案}、
\textbf{跨层视频帧权重差分技术}、\textbf{路径质量敏感的动态切换阈值算法}。三个创新点围绕提升Mesh网络整体
QoS展开。子网信道隔离的目的是在网络规划层面优化整体网络性能;视频帧优先级队列映射的作用
在于对视频帧进行不同的权重分解,以保证高权重的数据包能够优先传送;移动场景下的QoS保障
解决的问题是增强协议对链路质量变化的敏感度,从而缩减漫游中的路径切换时延,同时优化路由震荡
的问题。

第四章分三部分详细介绍了三个核心创新点在实际无线Mesh网络系统中如何设计实现。\textbf{
子网信道隔离}
通过在5GHz频段选取五个相互正交的信道,使得相邻子网之间相互隔离,并在Mesh子网上层架设一层
无线桥接网络。这样的实现不仅充分利用了5GHz频段的频谱资源,还将大的网络切割为较小的Mesh子网
,这种模块化的思想为后期的网络管理与维护及网络规模的扩建提供了支撑。\textbf{视频帧优先级
队列映射}设计了一个跨层的优先级队列映射机制。视频源采用GOP分层编码,第一跳节点计算每一个
视频帧数据包的权重,计算过程结合帧的类型、帧的位置以及是否为帧的头数据包。计算完成后将
权重值存储在网络层包头的ToS字段,后继节点不需要计算转发数据包的权重值。之后在发送阶段映射
模块会根据权重值将数据包插入合适的队列。\textbf{移动场景下的QoS保障}针对节点漫游时可能产生
的路径切换时延,重新调整了原BATMAN-adv协议的TQ值滑动窗口和OGM发包频率。并设计实现了动态
阈值标量的计算模块,从而避免了因为提神链路质量敏感度导致的路由震荡。在介绍三个创新点的设计
实现中贯穿了大量的时延以验证不同的功能带来的性能上的提升。

最终的实现的系统相对于之前的原始系统在整体系统性能上达到极大的提升。子网信道隔离使得实验
Mesh网络的性能提升达到10倍以上。视频真优先级队列映射针对视频质量度量指标PSNR的相对
于传统EDCA方式提升分别在50\%以上。移动场景下的QoS保障使得短路径到长路径的切换时延从30秒
缩减到4秒左右,并有效抑制了路由震荡。

\section{工作展望}
本文的工作部分解决了目前工业界大规模无线Mesh网络应用中的QoS保障的问题,单仍然存在很多可以
优化和改善的地方。比如子网信道隔离,虽然充分保障了子网之间的相互零干扰,但是占用了过多的信道
资源,可能会干扰其他非Mesh网络设备的正常工作,这一点我们在伊拉克实际部署时就曾经遭遇过。
如何在占用更少信道资源的情况下保障干扰最小化是一个值得研究的课题,它的核心其实就是信道分配
算法,学术界对这个问题也有过很多的研究工作。再如视频帧优先级队列映射,引入了802.11e的EDCA
机制,尝试了新的权重计算方法和映射机制,但是也还存在一定的局限性,比如现阶段只适用于视频
监控等少数场景,另外多跳路径上不同节点汇入的数据如何进行负载均衡也是一个值得研究的课题。
最后移动场景下的QoS保障方面,本文做到了在路径切换延时的大幅优化和路由震荡抑制,但是显然
4秒的延迟对于时延敏感的数据服务仍然存在巨大的提升空间,同时路由震荡本文的方法抑制效果在
前期并不理想,是否存在类似毒性逆转等传统网络中使用的方法可以更好的解决此问题也是值得
进一步研究的。总之,本文的工作取得了显著的效果,但还存在大量的问题值得研究。

我们也看到,现在无线Mesh网络技术迅猛,国内外很多公司都加入Mesh网络研发的队伍,比如华为、
微软、思科、Aruba、Strix等。学术界也一直在
研究其中的路由、安全、QoS保障等方面的问题,典型的项目有MIT的RoofNet项目、约翰霍普金斯大学的
SMesh项目等。IEEE802.11工作组几年前已经成立了Mesh网络工作组802.11s,该标准目前正在稳步推进。
随着相关的研发工作的投入,有理由相信无线Mesh网络将在人们的日常生产生活中发挥越来越重要
的作用。

