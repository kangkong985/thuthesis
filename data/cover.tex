\thusetup{
  %******************************
  % 注意:
  %   1. 配置里面不要出现空行
  %   2. 不需要的配置信息可以删除
  %******************************
  %
  %=====
  % 秘级
  %=====
  secretlevel={绝密},
  secretyear={2100},
  %
  %=========
  % 中文信息
  %=========
  ctitle={无线Mesh视频传输网络QoS优化技术研究},
  cdegree={工程硕士},
  cdepartment={软件学院},
  cmajor={软件工程},
  cauthor={王锋},
  csupervisor={刘云浩教授},
  cassosupervisor={**教授}, % 副指导老师
  % 日期自动使用当前时间,若需指定按如下方式修改:
  % cdate={超新星纪元},
  %
  % 博士后专有部分
  cfirstdiscipline={计算机科学与技术},
  cseconddiscipline={系统结构},
  postdoctordate={2009年7月——2011年7月},
  id={编号}, % 可以留空: id={},
  udc={UDC}, % 可以留空
  catalognumber={分类号}, % 可以留空
  %
  %=========
  % 英文信息
  %=========
  etitle={Research of QoS improvement in wireless Mesh network for video transmission},
  % 这块比较复杂,需要分情况讨论:
  % 1. 学术型硕士
  %    edegree:必须为Master of Arts或Master of Science(注意大小写)
  %             “哲学、文学、历史学、法学、教育学、艺术学门类,公共管理学科
  %              填写Master of Arts,其它填写Master of Science”
  %    emajor:“获得一级学科授权的学科填写一级学科名称,其它填写二级学科名称”
  % 2. 专业型硕士
  %    edegree:“填写专业学位英文名称全称”
  %    emajor:“工程硕士填写工程领域,其它专业学位不填写此项”
  % 3. 学术型博士
  %    edegree:Doctor of Philosophy(注意大小写)
  %    emajor:“获得一级学科授权的学科填写一级学科名称,其它填写二级学科名称”
  % 4. 专业型博士
  %    edegree:“填写专业学位英文名称全称”
  %    emajor:不填写此项
  edegree={Master of Software Engineering},
  emajor={Software Engineering},
  eauthor={Wang Feng},
  esupervisor={Professor Liu yunhao},
  eassosupervisor={},
  % 日期自动生成,若需指定按如下方式修改:
  % edate={December, 2005}
  %
  % 关键词用“英文逗号”分割
  ckeywords={Mesh网络, 视频监控, QoS, 正交信道},
  ekeywords={Mesh Network, video surveillance, QoS, orthogonal channel}
}

% 定义中英文摘要和关键字
\begin{cabstract}
  无线Mesh网络作为一种新兴的无线网络技术,以其特有的优势正在对人们的生产生活产生
  日益显著的影响。传统的AP网络设备,在为移动设备提供无线接口的同时,需要以有
  线形式接入外部网络,极大的限制了AP网络的灵活性和覆盖范围。另一种更加泛在的无线
  网络-3G/4G网络,虽然能够提供足够的网络覆盖范围,但网络基础设施部署和维护成本极
  高。在这样的背景下,无线Mesh网络以其优秀的灵活性、易部署等特性得到了广泛的尝试。
  目前,无线Mesh网络在校园、城市、野外、救灾、油田等很多场景下得到积极的推广和应
  用。但在提供便捷、灵活的网络接入方式的同时,也因其高度动态的路由机制带来了很多网
  络应用上的不足,比如无线链路的不稳定性,严重的信号间干扰,整体QoS无法保障等。
  本文的主要工作集中在解决无线Mesh网络在视频监控场景下的QoS保障所面临的挑战,所提
  出的技术方案在无线Mesh网络实际部署中亦具有指导意义。

  本文的创新点主要有:
  \begin{itemize}
    \item 大规模无线Mesh网络实践;
    \item 跨层的QoS保障技术。
  \end{itemize}

\end{cabstract}

% 如果习惯关键字跟在摘要文字后面,可以用直接命令来设置,如下:
% \ckeywords{\TeX, \LaTeX, CJK, 模板, 论文}

\begin{eabstract}
   An abstract of a dissertation is a summary and extraction of research work
   and contributions. Included in an abstract should be description of research
   topic and research objective, brief introduction to methodology and research
   process, and summarization of conclusion and contributions of the
   research. An abstract should be characterized by independence and clarity and
   carry identical information with the dissertation. It should be such that the
   general idea and major contributions of the dissertation are conveyed without
   reading the dissertation.

   An abstract should be concise and to the point. It is a misunderstanding to
   make an abstract an outline of the dissertation and words ``the first
   chapter'', ``the second chapter'' and the like should be avoided in the
   abstract.

   Key words are terms used in a dissertation for indexing, reflecting core
   information of the dissertation. An abstract may contain a maximum of 5 key
   words, with semi-colons used in between to separate one another.
\end{eabstract}

% \ekeywords{\TeX, \LaTeX, CJK, template, thesis}
