\thusetup{
  %******************************
  % 注意:
  %   1. 配置里面不要出现空行
  %   2. 不需要的配置信息可以删除
  %******************************
  %
  %=====
  % 秘级
  %=====
  secretlevel={绝密},
  secretyear={2100},
  %
  %=========
  % 中文信息
  %=========
  ctitle={无线Mesh视频传输网络QoS优化技术研究},
  cdegree={工程硕士},
  cdepartment={软件学院},
  cmajor={软件工程},
  cauthor={王锋},
  csupervisor={刘云浩教授},
  % cassosupervisor={**教授}, % 副指导老师
  % 日期自动使用当前时间,若需指定按如下方式修改:
  % cdate={超新星纪元},
  %
  % 博士后专有部分
  cfirstdiscipline={计算机科学与技术},
  cseconddiscipline={系统结构},
  postdoctordate={2009年7月——2011年7月},
  id={编号}, % 可以留空: id={},
  udc={UDC}, % 可以留空
  catalognumber={分类号}, % 可以留空
  %
  %=========
  % 英文信息
  %=========
  etitle={Research of QoS improvement in Wireless Mesh Network for video transmission},
  % 这块比较复杂,需要分情况讨论:
  % 1. 学术型硕士
  %    edegree:必须为Master of Arts或Master of Science(注意大小写)
  %             “哲学、文学、历史学、法学、教育学、艺术学门类,公共管理学科
  %              填写Master of Arts,其它填写Master of Science”
  %    emajor:“获得一级学科授权的学科填写一级学科名称,其它填写二级学科名称”
  % 2. 专业型硕士
  %    edegree:“填写专业学位英文名称全称”
  %    emajor:“工程硕士填写工程领域,其它专业学位不填写此项”
  % 3. 学术型博士
  %    edegree:Doctor of Philosophy(注意大小写)
  %    emajor:“获得一级学科授权的学科填写一级学科名称,其它填写二级学科名称”
  % 4. 专业型博士
  %    edegree:“填写专业学位英文名称全称”
  %    emajor:不填写此项
  edegree={Master of Software Engineering},
  emajor={Software Engineering},
  eauthor={Wang Feng},
  esupervisor={Professor Liu yunhao},
  eassosupervisor={},
  % 日期自动生成,若需指定按如下方式修改:
  % edate={December, 2005}
  %
  % 关键词用“英文逗号”分割
  ckeywords={Mesh网络, 视频监控, QoS, 正交信道},
  ekeywords={Mesh Network, video surveillance, QoS, orthogonal channel}
}

% 定义中英文摘要和关键字
\begin{cabstract}
  无线Mesh网络作为一种新兴的无线网络技术,以其特有的优势正在对人们的生产生活产生
  日益显著的影响。传统的AP网络设备,在为移动设备提供无线接口的同时,需要以有
  线形式接入外部网络,极大的限制了AP网络的灵活性和覆盖范围。另一种更加泛在的无线
  网络-3G/4G网络,虽然能够提供足够的网络覆盖范围,但网络基础设施部署和维护成本极
  高。在这样的背景下,无线Mesh网络以其优秀的灵活性、易部署等特性吸引了人们的兴趣。
  目前,无线Mesh网络在校园、城市、野外、救灾、油田等很多场景下得到积极的推广和应
  用。但在提供便捷、灵活的网络接入方式的同时,也因其高度动态的路由机制带来了很多网
  络应用上的不足,比如无线链路的不稳定性,移动链路干扰,网络拓扑不稳定等。无线Mesh
  网络在整体协议栈上缺乏QoS保障机制,MAC技术依赖802.11系列协议设定,但又无法直接引入
  802.11e的QoS保障技术。

  本文的主要工作是针对无线Mesh网络在视频监控场景下的QoS保障缺陷,提出了新的网络规划部署
  方案和跨层的QoS保障技术,使得100个节点的较大规模无线Mesh视频传输网络性能大幅提升,
  能够同时支撑200路高清视频传输,视频传输质量相对于原始Mesh网络在PSNR指标上提升50\%
  以上,动态场景下的路由切换延时由30秒降低到4秒,且有效抑制了网络路由震荡的问题。

  本文的创新点主要有:
  \begin{itemize}
    \item 探索了大规模无线Mesh视频监控网络部署的性能瓶颈,并提出了子网信道隔离的解决方案;
    \item 提出了跨层的QoS保障技术,包括跨层视频帧权重差分技术和路径质量敏感的动态切换阈值算法。
  \end{itemize}

\end{cabstract}

% 如果习惯关键字跟在摘要文字后面,可以用直接命令来设置,如下:
% \ckeywords{\TeX, \LaTeX, CJK, 模板, 论文}

\begin{eabstract}
Wireless Mesh Network(WMN), as an emerging wireless network technology, has exerted
increasingly huge impact on our daily life. In traditional AP wireless network,
AP routers provide wireless interface to mobile clients, while at the same time 
wired connection need to be provided for access to the external network, which 
will greatly limit the flexibility and coverage of AP network. Another more
ubiquitous wireless network is 3G/4G network. This kind of wireless access provide
much larger coverage to mobile clients. However, the cost for deployment and 
maintenance of the fundamental infrastructure is very high. Under this background, 
wireless Mesh network has attracted many peoples' interest with its' features
such as flexibility, self-organization, etc. Nowadays, some 
wireless mesh network projects has been deployed in school yard, rural areas, 
oil field, etc. On one hand, wireless mesh network bring with a lot of convenience.
On the other hand, many problems come with the highly dynamic routing mechanism
of WMN, like link instability, mobile interference, network topology instability 
and so on. WMN is lack of QoS support mechnism. Although the MAC layer inherits from 
IEEE802.11 standard, the QoS support mechnism of 802.11e will not be supported in
ad hoc network.

The main work of this paper will focus on working out an efficient approach
to alleviate the problem of lack of QoS support in the state-of-art WMN for 
video surveillance. We have proposed a new network planning and deployment scheme, 
and cross-layer QoS support technology. Experiments in actually 
deployed WMN system show that, our new scheme have significantly imporve the 
performance with the ability to support 200 way high-definition camera transmitting
over a large scale WMN consisted of 100 Wireless Mesh Nodes. On the other hand, 
the PSNR indicator has impoved more than 50\%, the time delay in dynamic scenario
has decreased from 30s to 4s.

The main innovation points are as follows:
\begin{itemize}
\item Explorint the bottle neck of the practical deployment of large scale WMN for surveillance.
and proposed the subnet channel isolation scheme.
\item Proposed a new cross-layer QoS support technology, including video
frame priority difference technologh and route quality aware dyamic switch threshold
algorithm.
\end{itemize}
\end{eabstract}

%\ekeywords{WMN, video surveillance, QoS, Orthogonal channel}
