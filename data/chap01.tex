\chapter{引言}
\label{cha:intro}

\section{背景及意义}
自1970年第一次无线数据通信演示始~\cite{IEEE80211},
无线设备迅速渗透到人们的生产生活的方方面面,小到嵌入式微处理器、移动设备,
大到交通工具、航空航天设备都囊括在内。在人们的日常生活中,最常见的无线网络
就是AP网络和3G/4G以及正在兴起的5G网络,前者提供有限覆盖范围和低灵活度的网络接入,但优点是部署成本低廉
,且网络性能稳定;后者极大的扩展了无线网络的覆盖范围,方便用户使用,
但也存在部署和维护成本高的缺点。
如今伴随着工业4.0的兴起,越来越多的传统设备智能化,另一方面智能终端越来越普及,人均
入网设备快速增长。无线网络的接入需求正在以惊人的速度增长。这就带来迅速增长的无线接入
需求和不完善的无线网络支撑体系之间的矛盾。面对这一困境,
无线Mesh网络作为一种可能的无线网络解决方案受到工业界的重视。相对于传统的无线接入方式,
无线Mesh网络具有自组织、自治愈、网络拓扑灵活、部署成本低廉等特性。这些特性使得无线Mesh
网络在灾后恢复、缺少有线基础设施的大型建筑物或者区域、蜂窝网络无法良好覆盖的区域等场景
下具有无可比拟的优势。

基于上述背景,本文工作的主要内容为基于无线Mesh网络设计实现部署灵活、低成本、高性能的大型
工业视频监控系统,该系统将在缺乏网络基础设施的野外油田等项目中部署并长期运行。
本文工作对无线mesh网络系统的大规模部署应用提供实际的借鉴意义,探索了其中QoS保障、
性能优化、网络整体规划等核心问题,并提出有效的优化方案。
同时对于工业视频监控网络提供一种新的高效的设计思路。

\section{主要工作}
本文基于室内大型实验床和室外实际运行的无线Mesh网络系统进行了大量的实验,在实验过程中
发现已有系统在QoS保障方面的不足,并针对这些问题设计实现优化方案。
同时开发了一系列
网络管理配置相关的工具,并深入Mesh网络各层进行以提升系统整体QoS为目标的优化。
本文主要工作现罗列如下:
\newline {1}.采用业界较为广泛使用的一种无线Mesh网络协议--batman-adv,基于linux
系统和Mikrotik硬件平台配置实际运行的Mesh节点设备,在此基础上搭建规模化的无线Mesh网络系统。
\newline {2}.以1为基础,进行大量实验,从实验中发现已有Mesh网络在QoS保障方面的不足。
归纳为三点:无线信道的开发性导致的无法规模化部署、视频监控场景下的视频大数据导致的网络
带宽紧张、设备移动时路径切换时延过长。
\newline {3}.针对2中的问题设计实现子网信道隔离的部署方案、跨层视频帧权重差分技术、
路径质量敏感的动态切换阈值算法。
\newline {4}.在伊拉克哈法亚油田等实际项目中部署项目系统,长期运行并监控维护。

\section{文章组织结构}
本文第一章引言部分介绍文章的背景和意义,并介绍文章的主要工作和贡献。第二章介绍该领
域的相关工作,面临的挑战。第三章介绍项目软硬件平台、项目设计和相关的知识。
第四章详细叙述项目中的三个核心创新点的设计实现及带来的性能提升。第五章总结全文并
展望后续工作。


